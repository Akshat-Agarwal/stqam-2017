\documentclass[11pt]{article}
\usepackage{listings}
\usepackage{tikz}
%\usepackage{algorithm2e}
\usetikzlibrary{arrows,automata,shapes}
\tikzstyle{block} = [rectangle, draw, fill=blue!20, 
    text width=5em, text centered, rounded corners, minimum height=2em]
\tikzstyle{bt} = [rectangle, draw, fill=blue!20, 
    text width=1em, text centered, rounded corners, minimum height=2em]

\newtheorem{defn}{Definition}
\newtheorem{crit}{Criterion}
\newcommand{\true}{\mbox{\sf true}}
\newcommand{\false}{\mbox{\sf false}}

\newcommand{\handout}[5]{
  \noindent
  \begin{center}
  \framebox{
    \vbox{
      \hbox to 5.78in { {\bf Software Testing, Quality Assurance and Maintenance } \hfill #2 }
      \vspace{4mm}
      \hbox to 5.78in { {\Large \hfill #5  \hfill} }
      \vspace{2mm}
      \hbox to 5.78in { {\em #3 \hfill #4} }
    }
  }
  \end{center}
  \vspace*{4mm}
}

\newcommand{\lecture}[4]{\handout{#1}{#2}{#3}{#4}{Lecture #1}}
\topmargin 0pt
\advance \topmargin by -\headheight
\advance \topmargin by -\headsep
\textheight 8.9in
\oddsidemargin 0pt
\evensidemargin \oddsidemargin
\marginparwidth 0.5in
\textwidth 6.5in

\parindent 0in
\parskip 1.5ex
%\renewcommand{\baselinestretch}{1.25}

\begin{document}

\lecture{31 --- March 23, 2015}{Winter 2015}{Patrick Lam}{version 0.5}

\section*{Functionality-based Input Domain Modelling}
Use the intended functionality of system to create tests.

\paragraph{containsElement example.} Consider again the
method {\tt containsElement}, 
which takes a list {\tt l} and an element {\tt e}. It behaves as
follows. If {\tt l} or {\tt e} is null, throw a {\tt
  NullPointerException}. Otherwise, return {\tt true} if {\tt l}
contains {\tt e} and {\tt false} otherwise.

Some possible characteristics:\\[4em]
% number of occurences of e in l: 0, 1, > 1
% element occurs first in list: t, f

Notes:
\begin{itemize}
\item (+) might yield better test cases due to domain knowledge;
\item (+) can create these models from specifications; don't need implementations;
\item (-) may be hard to identify values and characeristics;
\item (-) harder to generate tests from such an IDM.
\end{itemize}

\section*{Identifying Characteristics}
Here are some possible sources of characteristics for functionality-based IDMs:
\begin{itemize}
\item preconditions and postconditions. e.g. 1) {\tt Object.wait()}
  precondition: must hold lock; characteristic: lock held or not (the state
includes the locks currently held). 2) {\tt TriTyp} triangle classification;
we might construct characteristics based on the return value.
\item relationships between variables, e.g. parameter aliasing;
\item specifications contain more concentrated domain knowledge, so it is
easier to extract characteristics from them, if they are available.
\end{itemize}

Using fewer blocks usually gives a more tractable partition, and
it is easier to ensure that the partition is disjoint and complete.

\section*{Choosing Blocks and Values}
This part is the most creative in input space testing. Here are some tips
for finding good values to test with. 

Be sure to include:
\begin{itemize}
\item valid values and invalid values;
\item boundary values and non-boundary values;
\item special values (0, {\tt null}, empty set);
\end{itemize}
and choose your blocks and partitions appropriately; if you need more
partitions, you can divide existing partitions, especially those containing
valid values. Check that partitions are disjoint and complete.

\paragraph{Trityp Example.} We discuss the standard triangle classification
example again, which takes three side lengths as input.

Interface-based try 1:

\begin{center}
\begin{tabular}{lccc}
side 1 is: & $ > 0$ & $\qquad < 0$ & $\qquad= 0$ \\
side 2 is: & $ > 0$ & $\qquad < 0$ & $\qquad= 0$ \\
side 3 is: & $ > 0$ & $\qquad < 0$ & $\qquad= 0$ \\
\end{tabular}
\end{center}

We might refine some partitions, assuming that we take integers as parameters:

\begin{center}
\begin{tabular}{lcccc}
side 1 is: & $ > 1$ & $\qquad 1 \qquad$ & 0 & $\qquad < 0$ \\
side 2 is: & $ > 1$ & $\qquad 1 \qquad$ & 0 & $\qquad < 0$ \\
side 3 is: & $ > 1$ & $\qquad 1 \qquad$ & 0 & $\qquad < 0$ \\
\end{tabular}
\end{center}

Generating tests for these partitions is straightforward.

Functionality-based try (wrong):

\begin{center}
scalene $\qquad$ isoceles $\qquad$ equilateral $\qquad$ invalid 
\end{center}

Unfortunately, equilateral triangles are also isoceles, so we should instead use:

\begin{center}
scalene $\qquad$ isoceles, not equilateral $\qquad$ equilateral $\qquad$ invalid 
\end{center}

Observe that it's not completely straightforward to generate tests
from functionality-based partitions. We can, however, generate the following tests:

\begin{center}
$(4, 5, 6)$ $\qquad$ $(3, 3, 4)$ $\qquad$ $(3,3,3)$ $\qquad$ $(3, 4, 8)$
\end{center}

\paragraph{Alternate approach.} Instead of multiple blocks (like 4 in the above
example), use T/F characteristics (e.g. isEquilateral, etc). Then
disjointness and completeness are guaranteed, but you get more characteristics.

\paragraph{Multiple IDMs.} We can use one IDM for valid values and
a second IDM for invalid values, then differentially refine or cover
these IDMs.

\paragraph{Verifying IDMs.} Some tips:

\begin{itemize}
\item ``Are we missing anything we could include?''
\item check completeness and disjointness; if using multiple IDMs, only the
sum of the IDMs needs to be complete.
\end{itemize}

\end{document}
