\documentclass[11pt]{article}
\usepackage{listings}
\usepackage{tikz}
\usepackage{enumerate}
\usepackage{url}
\usepackage{amssymb}
\usetikzlibrary{arrows,automata,shapes}

\lstset{basicstyle=\ttfamily \scriptsize,
  basicstyle=\ttfamily,
   columns=fullflexible,
   breaklines=true,
   numbers=left,
   numberstyle=\scriptsize,
   stepnumber=1,
   mathescape=false,
   tabsize=2,
   showstringspaces=false
}
\newtheorem{defn}{Definition}
\newtheorem{crit}{Criterion}

\newcommand{\handout}[5]{
  \noindent
  \begin{center}
  \framebox{
    \vbox{
      \hbox to 5.78in { {\bf Software Testing, Quality Assurance and Maintenance } \hfill #2 }
      \vspace{4mm}
      \hbox to 5.78in { {\Large \hfill #5  \hfill} }
      \vspace{2mm}
      \hbox to 5.78in { {\em #3 \hfill #4} }
    }
  }
  \end{center}
  \vspace*{4mm}
}

\newcommand{\lecture}[4]{\handout{#1}{#2}{#3}{#4}{Lecture #1}}
% 1-inch margins, from fullpage.sty by H.Partl, Version 2, Dec. 15, 1988.
\topmargin 0pt
\advance \topmargin by -\headheight
\advance \topmargin by -\headsep
\textheight 8.9in
\oddsidemargin 0pt
\evensidemargin \oddsidemargin
\marginparwidth 0.5in
\textwidth 6.5in

\parindent 0in
\parskip 1.5ex
%\renewcommand{\baselinestretch}{1.25}

% http://gurmeet.net/2008/09/20/latex-tips-n-tricks-for-conference-papers/
\newcommand{\squishlist}{
 \begin{list}{$\bullet$}
  { \setlength{\itemsep}{0pt}
     \setlength{\parsep}{3pt}
     \setlength{\topsep}{3pt}
     \setlength{\partopsep}{0pt}
     \setlength{\leftmargin}{1.5em}
     \setlength{\labelwidth}{1em}
     \setlength{\labelsep}{0.5em} } }
\newcommand{\squishlisttwo}{
 \begin{list}{$\bullet$}
  { \setlength{\itemsep}{0pt}
     \setlength{\parsep}{0pt}
    \setlength{\topsep}{0pt}
    \setlength{\partopsep}{0pt}
    \setlength{\leftmargin}{2em}
    \setlength{\labelwidth}{1.5em}
    \setlength{\labelsep}{0.5em} } }
\newcommand{\squishend}{
  \end{list}  }

\lstdefinelanguage{JavaScript}{
  keywords={typeof, new, true, false, catch, function, return, null, catch, switch, var, if, in, while, do, else, case, break},
  keywordstyle=\color{blue}\bfseries,
  ndkeywords={class, export, boolean, throw, implements, import, this},
  ndkeywordstyle=\color{darkgray}\bfseries,
  identifierstyle=\color{black},
  sensitive=false,
  comment=[l]{//},
  morecomment=[s]{/*}{*/},
  commentstyle=\color{purple}\ttfamily,
  stringstyle=\color{red}\ttfamily,
  morestring=[b]',
  morestring=[b]''
}


\begin{document}

\lecture{14 --- February 4, 2015}{Winter 2015}{Patrick Lam}{version 1}

Recall that we've been discussing beliefs. Here are a couple of beliefs
that are worthwhile to check. (examples courtesy Dawson Engler.)

\paragraph{Redundancy Checking.} 1) Code ought to do something. So,
when you have code that doesn't do anything, that's suspicious. Look
for identity operations, e.g.
\[ {\tt x = x}, ~ {\tt 1 * y}, ~ {\tt x \& x}, ~ {\tt x | x}. \]
Or, a longer example:
\begin{lstlisting}[language=C]
    /* 2.4.5-ac8/net/appletalk/aarp.c */
    da.s_node = sa.s_node;
    da.s_net = da.s_net;
\end{lstlisting}

Also, look for unread writes:
\begin{lstlisting}[language=C]
    for (entry=priv->lec_arp_tables[i]; 
         entry != NULL; entry=next) {
      next = entry->next; // never read!
      ...
    }
\end{lstlisting} 
Redundancy suggests conceptual confusion.


%\section*{Inferring beliefs}
%If we are to verify beliefs about the code (either MAY or MUST), we first need to get a set of beliefs somehow.


So far, we've talked about MUST-beliefs; violations are clearly wrong (in some sense).
Let's examine MAY beliefs next. For such beliefs, we need more evidence to convict the program.

\paragraph{Process for verifying MAY beliefs.} We proceed as follows:
\begin{enumerate}
\item    Record every successful MAY-belief check as ``check''.
\item    Record every unsucessful belief check as ``error''.
\item    Rank errors based on ``check'' : ``error'' ratio.
\end{enumerate}
Most likely errors occur when ``check'' is large, ``error'' small.

\paragraph{Example.} One example of a belief is use-after-free: 
 \begin{lstlisting}[language=C]
    free(p);
    print(*p);
 \end{lstlisting}
That particular case is a MUST-belief. 
However, other resources are freed by custom (undocumented) free functions.
It's hard to get a list of what is a free function and what isn't.
So, let's derive them behaviourally.

\paragraph{Inferring beliefs: finding custom free functions.}
The key idea is:
    if pointer {\tt p} is not used after calling {\tt foo(p)},
    then derive a MAY belief that {\tt foo(p)} frees {\tt p}.

OK, so which functions are free functions? Well, just assume all functions free all arguments:
\begin{itemize}
\item emit ``check'' at every call site;
\item emit ``error'' at every use.
\end{itemize}
(in reality, filter functions with suggestive names).

Putting that into practice,
we might observe:

\begin{center}
\begin{tabular}{l|l|l|l|l|l}
\begin{minipage}{5em}
foo(p)\\
*p = x;
\end{minipage} &
\begin{minipage}{5em}
foo(p)\\
*p = x;
\end{minipage} &
\begin{minipage}{5em}
foo(p)\\
*p = x;
\end{minipage} &
\begin{minipage}{5em}
bar(p)\\
p = 0;
\end{minipage} &
\begin{minipage}{5em}
bar(p)\\
p=0;
\end{minipage} &
\begin{minipage}{5em}
bar(p)\\
*p = x;
\end{minipage} 
\end{tabular}
\end{center}

We would then rank {\tt bar}'s error first.
Plausible results might be: 23 free errors, 11 false positives.


\paragraph{Inferring beliefs: finding routines that may return {\tt NULL}.}
The situation: we want to know which routines may return {\tt NULL}.
Can we use static analysis to find out?
\squishlist
      \item sadly, this is difficult to know statically (``{\tt return p->next;}''?) and,
      \item we get false positives: some functions return {\tt NULL} under special cases only.
\squishend

Instead, let's observe what the programmer does.
Again, rank errors based on checks vs non-checks.
As a first approximation, assume {\bf all} functions can return {\tt NULL}.
\squishlist
  \item if pointer checked before use: emit ``check'';
  \item if pointer used before check: emit ``error''.
\squishend

This time, we might observe:

\begin{center}
\begin{tabular}{l|l|l|l}
\begin{minipage}{6em}
p = bar(...);\\
*p = x;
\end{minipage} &
\begin{minipage}{7em}
p = bar(...);\\
if (!p) return;\\
*p = x;
\end{minipage} &\begin{minipage}{7em}
p = bar(...);\\
if (!p) return;\\
*p = x;
\end{minipage} &\begin{minipage}{7em}
p = bar(...);\\
if (!p) return;\\
*p = x;
\end{minipage}
\end{tabular}
\end{center}

Again, sort errors based on the ``check'':``error'' ratio.

Plausible results: 152 free errors, 16 false positives.

\newpage
\subsection*{General statistical technique}
When we write ``a(); \ldots b();'', we mean a MAY-belief that a() is followed by b().
We don't actually know that this is a valid belief. It's a hypothesis, and we'll try it out.
Algorithm: 
\vspace*{-1em}
\squishlist
\item assume every {\tt a}--{\tt b} is a valid pair;
\item emit ``check'' for each path with ``a()'' and then ``b()'';
\item emit ``error'' for each path with ``a()'' and no ``b()''.
\squishend
(actually, prefilter functions that look paired).

Consider:

{\small
\begin{tabular}{l|l|l}
\begin{minipage}{10em}
foo(p, \ldots);\\
bar(p, \ldots); // check 
\end{minipage} &
\begin{minipage}{10em}
foo(p, \ldots);\\
bar(p, \ldots); // check 
\end{minipage} &
\begin{minipage}{12em}
foo(p, \ldots);\\
// error: foo, no bar!
\end{minipage}
\end{tabular}
}

This applies to the course project as well.

\begin{tabular}{ll}
\begin{minipage}{10em}
\scriptsize
\begin{lstlisting}
void scope1() {
 A(); B(); C(); D();
}

void scope2() {
 A(); C(); D();
}

void scope3() {
 A(); B();
}

void scope4() {
 B(); D(); scope1();
}

void scope5() {
 B(); D(); A();
}

void scope6() {
 B(); D();
}
\end{lstlisting}
\end{minipage} &\begin{minipage}{30em}
``A() and B() must be paired'':\\
either A() then B() or B() then A().\\[2em]
\begin{tabbing}
{\bf Support} =  \= \# times a pair of functions appears together.\\
\> \hspace*{2em} support(\{A,B\})=3
\end{tabbing}
~\\[1em]
{\bf Confidence(\{A,B\},\{A\}}) = \\ \hspace*{2em} support(\{A,B\})/support(\{A\}) = 3/4
\end{minipage}
\end{tabular}

Sample output for support threshold~3, confidence threshold 65\% (intra-procedural analysis):
{\small
\squishlist
\item bug:A in scope2, pair: (A B), support:~3, confidence: 75.00\%
\item bug:A in scope3, pair: (A D), support:~3, confidence: 75.00\%
\item bug:B in scope3, pair: (B D), support:~4, confidence: 80.00\%
\item bug:D in scope2, pair: (B D), support:~4, confidence: 80.00\%
\squishend
}

The point is to find examples like the one from {\tt cmpci.c}
where there's a {\tt lock\_kernel()} call, but, on an exceptional path, no
{\tt unlock\_kernel()} call.
\vspace*{-1em}

\paragraph{Summary: Belief Analysis.}
      We don't know what the right spec is.
      So, look for contradictions.

\squishlist
\item      MUST-beliefs: contradictions = errors!
\item      MAY-beliefs: pretend they're MUST, rank by confidence.
\squishend
(A key assumption behind this belief analysis technique: most of the code is correct.)

\paragraph{Further references.}
Dawson~R. Engler, David Yu Chen, Seth Hallem, Andy Chou and Benjamin Chelf.
``Bugs as Deviant Behaviors: A general approach to inferring errors in systems code''.
In SOSP '01.

Dawson~R. Engler, Benjamin Chelf, Andy Chou, and Seth Hallem.
``Checking system rules using system-specific, programmer-written
  compiler extensions''.
In OSDI '00 (best paper).
\url{www.stanford.edu/~engler/mc-osdi.pdf}

Junfeng Yang, Can Sar and Dawson Engler.
``eXplode: a Lightweight, General system for Finding Serious Storage System Errors''.
In OSDI'06.
\url{www.stanford.edu/~engler/explode-osdi06.pdf}

\section*{Using Linters}
We will also talk about linters in this lecture, based on Jamie Wong's blog post \url{jamie-wong.com/2015/02/02/linters-as-invariants/}.

\paragraph{First there was C.}
In statically-typed languages, like C,
\begin{lstlisting}[language=C]
#include <stdio.h>

int main() {
  printf("%d\n", num);
  return 0;
}
\end{lstlisting}
the compiler saves you from yourself.
The guaranteed invariant:
\begin{center}
``if code compiles, all symbols resolve.''
\end{center}

\paragraph{Less-nice languages.}
OK, so you try to run that in JavaScript and it crashes right away.
Invariant?
\begin{center}
 ``if code runs, all symbols resolve?''
\end{center}

But what about this:
\begin{lstlisting}[language=JavaScript]
function main(x) {
  if (x) {
    console.log("Yay");
  } else {
    console.log(num);
  }
}

main(true);
\end{lstlisting}
Nope! The above invariant doesn't work.

OK, what about this invariant:
\begin{center}
``if code runs without crashing, all symbols referenced in the code path executed resolve?''
\end{center}

Nope!
\begin{lstlisting}[language=JavaScript]
function main() {
  try {
    console.log(num);
  } catch (err) {
    console.log("nothing to see here");
  }
}

main();
\end{lstlisting}

So, when you're working in JavaScript and maintaining old code, you always have to
deduce:
\squishlist
\item is this variable defined?
\item is this variable always defined?
\item do I need to load a script to define that variable?
\squishend
We have computers. They're powerful. Why is this the developer's problem?!

\paragraph{Solution: Linters.}
\begin{lstlisting}[language=JavaScript]
function main(x) {
  if (x) {
    console.log("Yay");
  } else {
    console.log(num);
  }
}

main(true);
\end{lstlisting}

Now:
\begin{verbatim}
$ nodejs /usr/local/lib/node_modules/jshint/bin/jshint --config jshintrc foo.js
foo.js: line 5, col 17, 'num' is not defined.

1 error
\end{verbatim}

\vspace*{-1em}
\paragraph{Invariant:}~\\

\begin{center}
``If code passes JSHint, all top-level symbols resolve.''
\end{center}

\paragraph{Strengthening the Invariant.} Can we do better? How about adding a pre-commit hook?
\begin{center}
``If code is checked-in and commit hook ran,\\ all top-level symbols resolve.''
\end{center}
Of course, sometimes the commit hook didn't run. Better yet:
\squishlist
\item Block deploys on test failures.
\squishend

\paragraph{Better invariant.}~\\[1em]
\begin{center}
``If code is deployed,\\ all top-level symbols resolve.''
\end{center}

\paragraph{Even better yet.}
It is hard to tell whether code is deployed or not.
Use git feature branches, merge when deployed.
\begin{center}
``If code is in master,\\ all top-level symbols resolve.''
\end{center}

\end{document}
