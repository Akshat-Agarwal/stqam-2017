\documentclass{article}
\usepackage{fullpage}
\usepackage{url}
\usepackage{enumitem}
\setlist{nosep}

\begin{document}
\title{Software Testing, Quality Assurance and Maintenance\\SE465/CS447/ECE453, Winter 2017, version 1}
\author{Patrick Lam}
\renewcommand{\today}{}
\maketitle
\vspace*{-2em}

\section*{Brief Overview}
As you have no doubt discovered, software never works right from the
start. A key technique for getting more acceptable software is
testing. Organized testing can help identify problems in software
systems, enabling developers to fix these problems.  This course will
introduce software testing techniques; while it's not my goal
to produce testers, you should at least be conversant
with up-to-date testing methodologies and techniques. And easier-to-test
software is better software.

In this class, we will also touch on software maintenance. While we
greatly (over?) emphasize design in engineering school, maintenance
consumes a large fraction of today's software development resources.

\section*{General Information}

\noindent
\begin{tabbing}
  {\bf Course web page:~~~} \= \url{https://patricklam.ca/stqam}\\
  {\bf github repo:}\> \url{git@github.com:patricklam/stqam-2017.git}\\
  {\bf Lectures:} \> MWF 11:30-12:20, MC 4020 (SE465)\\
\> MWF 15:30-16:20, EIT 1015 (ECE453/CS447)\\
{\bf Tutorials:} \> not generally used, but scheduled for F 17:30-18:20, EIT 1015
\end{tabbing}

\noindent
{\bf Instructor:} \\

\noindent
\hspace*{2em} \begin{minipage}{.6\textwidth}
Prof. Patrick Lam\\
Office: DC2597C\\
Office Hours: M 12:30-13:30, or by appointment\\
Email: {\tt patrick.lam@uwaterloo.ca}\\
Phone: Use email instead!

\end{minipage} \\[1em]

\noindent {\bf Lab Instructor:} \\

\noindent
\hspace*{2em} \begin{minipage}{.6\textwidth}
Irene Huang\\
Email: {\tt yqhuang@uwaterloo.ca}\\

\end{minipage} \\[1em]


\noindent {\bf Teaching Assistants:} \\

\noindent
\hspace*{2em}\begin{tabular}{l@{\hspace*{3em}}l}
  Parsa Pourali&
  {\tt ppourali@uwaterloo.ca}\\
  Gursharan Sandhu&
  {\tt g27sandh@uwaterloo.ca}\\
  Mier Ta&
  {\tt m2ta@uwaterloo.ca}\\
  Xinye Tang&
  {\tt x58tang@uwaterloo.ca}\\
  Song Wang&
  {\tt s446wang@uwaterloo.ca}\\
  Jun Zhao&
  {\tt j283zhao@uwaterloo.ca}\\
\end{tabular}

\section*{Course Description}
\paragraph{Objectives.}
We hope that you will learn how to test as a developer, thus making
you a better developer. Good software systems are easy to test and
we'll talk about how testability affects design.
\begin{itemize}
  \item You will be able to create and evaluate test suites for reasonably-sized
software systems.

  \item You will learn how to use and write tools for software maintenance and
verification (particularly automated testing tools).
\end{itemize}

\paragraph{Topics.}
Over the past few years, we've been working to improve this course.
Be aware that, when studying for exams, this year's offering differs
significantly from previous offerings. Near exam time, I'll indicate
the types of exam questions that will still be relevant and those that
won't be.~\\

\begin{itemize}
\item Introduction and definitions (defects, faults, failures) (1 week)
\item \emph{Defining Test Suites}---finding interesting inputs and knowing when to stop looking (4 weeks)\\
How do you generate test suites?
\begin{itemize}
\item open-ended exploratory testing;
\item statement/branch coverage;
\item graph-based models of program state; integrating design documentations; 
\item automatically generating inputs: grammars, fuzzing.
\end{itemize}~\\[-1em]
How good is your test suite?
\begin{itemize}
\item coverage
\item mutation
\end{itemize}

\item \emph{Engineering Test Suites} (3 weeks)
\begin{itemize}
\item JUnit (unit testing, end-to-end testing, regression testing)
\item Selenium and other web-based testing
\item mock objects
\item refactoring tests
\item refactoring code to be testable
\item continuous integration
\item flaky tests
\end{itemize}

\item \emph{Tools for Verification \& Validation} (3 weeks)\\
I plan to include some intuition for why these tools work. List of tools may change.

\begin{itemize}
\item concept: static vs. dynamic analysis
\item static approaches: type systems, immutability, compiler warnings, software model checking, JML tools. Coverity, Facebook Infer, FindBugs, aComment
\item dynamic approaches: valgrind/Clang address sanitizer, assertions, concurrency detectors: races, missing locks
\item human-based approaches: code review, bug reporting
\end{itemize}

\item Bonus: debugging and the scientific method (1 week) \\
(reference: Andreas Zeller. \emph{Why Programs Fail: a Guide to Systematic Debugging.})
\end{itemize}

\section*{Reference Material}
Course notes (posted to the git repository) are your best bet.

\newpage
\section*{Evaluation}
This course includes assignments, a midterm, a course project, and a final
examination.

\begin{tabular}{lrl}
3 individual assignments & 20\% & (6 2/3\% each) \\
Course project (in groups, up to 3/group) & 15\% \\
Midterm & 15\% \\
Final exam & 50\% \\
\end{tabular}

\noindent The midterm and final exams will be open-book, open-notes. 

\paragraph{Schedule.} Assignment handin will be done through the git server at {\tt ecgit.uwaterloo.ca}.\\[-1em]
\begin{center}
\begin{tabular}{ll}
January 16	&A1 out\\
January 30	&A1 due, A2 out\\
February 17 &Midterm (6:30-8PM)\\
February 27	&A2 due, A3 out\\
March 24	&A3 due\\
April 3 	&Last lecture; project due\\
Exam period	&Final exam
\end{tabular}~\\
\end{center}
You can also find the dates on the following Google calendar:\\
\hspace*{3em}\url{juj4cn4vtkeivufdvd6iqo6h8o@group.calendar.google.com}

\section*{Policies}

\paragraph{Group work.} The project will be done in groups.
You may discuss assignments with others, but I expect each of
you to do the assignment independently. I will follow UW's Policy 71
if I discover any cases of plagiarism. I will not use turnitin.

\paragraph{Lateness.} You have 2 days of lateness to use on assignment 
submissions throughout the term. Each day you hand in an assignment
late consumes one of the days of lateness. If you consume all of your
late days, assignments that are still late will get 0. Missed assignments get 0.
e.g. you may hand in A1 one day late and A2 one day late if
you hand in A3 on time.  Or you can hand in A1-A2 on time and
A3 two days late.\\[1em]

\noindent \begin{tabular}{ @{\hspace{0.25in}}l l }
Academic integrity: & \url{http://uwaterloo.ca/academicintegrity/}\\
Petition \& Grievance:
& \url{http://secretariat.uwaterloo.ca/Policies/policy70.htm}\\
Discipline: & \url{http://secretariat.uwaterloo.ca/Policies/policy71.htm} \\
Penalties: 
&  \url{http://secretariat.uwaterloo.ca/guidelines/penaltyguidelines.htm}\\
Appeals: & \url{http://secretariat.uwaterloo.ca/Policies/policy72.htm} \\
AccessAbility: & \url{https://uwaterloo.ca/disability-services/}
\end{tabular}


\end{document}
