\documentclass[10pt,hidelinks]{article}
\usepackage[letterpaper, hmargin=0.75in, vmargin=0.75in]{geometry}
\usepackage{graphicx}
\usepackage[hyphens]{url}
\usepackage{hyperref}
\usepackage{listings}
\usepackage{pgf}
\usepackage{courier}
\usepackage{fancyvrb}

\usepackage{upquote}
\newcommand\textcode{\Verb}

\parindent 0in
\parskip 1.5ex


\lstset{ %
language=Java,
basicstyle=\ttfamily\scriptsize,commentstyle=\scriptsize\itshape,showstringspaces=false,breaklines=true}


\begin{document}

\title{
ECE453/CS447/SE465 \\
Software Testing, Quality Assurance, and Maintenance\\
Assignment/Lab 1: Using DigitalOcean}
\author{Patrick Lam}
\date{January 22, 2017}
\renewcommand{\today}{}
\maketitle

An alternative to using Vagrant is spinning up a virtual machine with a cloud provider.
I'll provide instructions for DigitalOcean here, but Lightsail (by Amazon) is an option also.
This does cost you some money (US\$10/month prorated while your VM is up), but always works.

\paragraph{Sign up with digitalocean.} Go to \url{digitalocean.com}, create an account, and give
them your billing information.

\paragraph{Create droplet.} The digitalocean term for a VM is a ``droplet''. Click on the
``Create droplet'' button top right. Choose ``Ubuntu 16.04.1 x64'' for the image, ``\$10/mo'' 
for the size (I don't think you can Dart with 512MB). You'll get better ping times with the
TOR1 datacenter.

The best practice is to add your SSH key. You can use the same one as you used with ecgit.
If you cut and paste the ssh key, be sure to include the ``ssh-rsa'' prefix at the beginning.

%138.197.76.38

Write down the IP address that digitalocean tells you (e.g. 138.197.76.38).

\section*{Initial setup: installing packages}

We'll next mirror the steps that {\tt bootstrap.sh} runs automatically. ssh to {\tt root@your-ip-address}. Run the following commands to install the JDK, Maven, Git and Dart.

\textcode!sh -c 'curl https://dl-ssl.google.com/linux/linux_signing_key.pub | apt-key add -'!\\
\textcode!sh -c 'curl https://storage.googleapis.com/download\!\\
\textcode!.dartlang.org/linux/debian/dart_stable.list > /etc/apt/sources.list.d/dart_stable.list'!\\
\textcode!apt-get update!\\
\textcode!apt-get install -y default-jdk maven git dart!

\section*{Creating a non-root user}
Best not to develop as root. Let's create a non-privileged account and set it up.

\begin{verbatim}
adduser (your-favorite-userid)
mkdir ~userid/.ssh
cp ~/.ssh/authorized_keys ~userid/.ssh/
echo "export PATH=/usr/lib/dart/bin:$PATH" > ~userid/.bash_profile
chown -R userid.userid ~userid/.ssh
chown userid.userid ~userid/.bash_profile
\end{verbatim}

\section*{Assignment 1 setup}
Next, logout and reconnect to your machine as your userid. Explicitly specify ``ssh -A'' to pass along your ssh key to the virtual machine.

Assuming that you've already forked the repo, clone it on your
virtual machine:

\begin{verbatim}
git clone git@ecgit.uwaterloo.ca:se465/1171/userid/a1
\end{verbatim}

Check that everything works by running ``mvn package'' from the ``a1/shared/bukkit'' directory and ``pub get; pub build'' from the ``a1/shared/average'' directory.

\section*{Fixing ``average''}
The ``average'' webapp fetches from the server at a hardcoded ``localhost'' location. That's not going to work.

Edit the file ``a1/shared/average/lib/client/student\_service.dart'' and replace ``localhost:8088'' with ``your-ip-address:8088'' in the constructor call to {\tt StudentService}.

\section*{Viewing results}
To view your coverage results, you can either serve the {\tt a1/shared/bukkit/target/site} directory from your DigitalOcean machine or you can rsync it to your local machine. Rsyncing is easier. At a prompt on your local machine:

\begin{verbatim}
rsync -avz your-ip-address:a1/shared/bukkit/target/site .
\end{verbatim}

You'll then find that you have a copy of the coverage report in your current directory.

Good luck!

\end{document}

