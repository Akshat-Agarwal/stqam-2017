\documentclass[11pt]{article}
\usepackage{enumitem}
\usepackage{listings}
\usepackage{tikz}
\usepackage{url}
%\usepackage{algorithm2e}
\usetikzlibrary{arrows,automata,shapes}
\tikzstyle{block} = [rectangle, draw, fill=blue!20, 
    text width=5em, text centered, rounded corners, minimum height=2em]
\tikzstyle{bt} = [rectangle, draw, fill=blue!20, 
    text width=1em, text centered, rounded corners, minimum height=2em]

\lstset{ %
language=Java,
basicstyle=\ttfamily,commentstyle=\scriptsize\itshape,showstringspaces=false,breaklines=true,numbers=left}

\newtheorem{defn}{Definition}
\newtheorem{crit}{Criterion}

\newcommand{\handout}[5]{
  \noindent
  \begin{center}
  \framebox{
    \vbox{
      \hbox to 5.78in { {\bf Software Testing, Quality Assurance and Maintenance } \hfill #2 }
      \vspace{4mm}
      \hbox to 5.78in { {\Large \hfill #5  \hfill} }
      \vspace{2mm}
      \hbox to 5.78in { {\em #3 \hfill #4} }
    }
  }
  \end{center}
  \vspace*{4mm}
}

\newcommand{\lecture}[4]{\handout{#1}{#2}{#3}{#4}{Lecture #1}}
\topmargin 0pt
\advance \topmargin by -\headheight
\advance \topmargin by -\headsep
\textheight 8.9in
\oddsidemargin 0pt
\evensidemargin \oddsidemargin
\marginparwidth 0.5in
\textwidth 6.5in

\parindent 0in
\parskip 1.5ex
%\renewcommand{\baselinestretch}{1.25}

\begin{document}

\lecture{5 --- January 13, 2017}{Winter 2017}{Patrick Lam}{version 1}

\paragraph{Last time.} We spent most of the time talking about the distinction between
static and dynamic analysis of code. Static analysis finds all the
things (perhaps too many), while dynamic analysis tells you exactly
what happens on a particular execution (but you have to have the right
inputs). We also looked at a JUnit test case
(setup/teardown/establishing state/checking system under test
actions); the test case we looked at was at
\url{https://github.com/google/guava/blob/master/guava-tests/test/com/google/common/io/MoreFilesTest.java}. Finally,
we introduced the notion of coverage and test requirements a bit more formally.


\section*{Exploratory Testing}

Exploratory testing is usually (but not always) carried out by
dedicated testers. In that sense, it's somewhat different from the
other testing activities in this course, which are more
developer-focussed---our usual goal is learning, as developers, how to
deploy better automated test suites for our software. Hallway
usability testing, though, is an application of exploratory testing.
Furthermore, the dedicated QA function is important, and we should
learn about how it works.

\paragraph{Resources.} James Bach has a shorter and a longer introduction 
to exploratory testing:\\[-2em]
\begin{itemize}[noitemsep]
\item \url{http://www.satisfice.com/articles/what_is_et.shtml}
\item \url{http://www.satisfice.com/articles/et-article.pdf}
\end{itemize}
There is an exhaustive set of notes on exploratory testing by Cem Kaner:\\[-2em]
\begin{itemize}
\item http://www.kaner.com/pdfs/QAIExploring.pdf
\end{itemize}

\begin{quote}
``Exploratory testing is simultaneous learning, test design, and test execution.''
\end{quote}
Contrast this to scripted testing, where test design
happens ahead of time and then test execution happens (repeatedly)
throughout the product's development cycle. Often, when we think of
dedicated QA teams, we think that they are manually executing scripted
tests. In 2017, that is not a terribly effective use of workers.

There is a continuum between scripted testing and exploratory testing.
Good exploratory testing may use prepared scripts for certain tasks.

\newpage
\paragraph{Scenarios where Exploratory Testing Excels.} (from Bach's article)
\begin{itemize}[noitemsep]
\item providing rapid feedback on new product/feature;
\item learning product quickly;
\item diversifying testing beyond scripts;
\item finding single most important bug in shortest time;
\item independent investigation of another tester's work;
\item investigating and isolating a particular defect;
\item investigate status of a particular risk to evaluate need for scripted tests.
\end{itemize}

\paragraph{Exploratory Testing Process.} Exploratory testing should not be
randomly bumbling around (we can call that ``ad hoc testing'')---the random
approach finds bugs but isn't the most efficient at giving you an idea of how well
the software works.

\begin{itemize}[noitemsep]
\item Start with a charter for your testing activity, e.g. ``Explore and analyze the product elements of the software.'' 
These charters should be somewhat ambiguous.
\item Decide what area of the software to test.
\item Design a test (informally).
\item Execute the test; log bugs.
\item Repeat.
\end{itemize}
Exploratory testing shouldn't produce an exhaustive set of notes. Good testers will
be able to reproduce the bugs that they encounter during their testing from brief notes.
Taking full notes takes too long.

The output from exploratory testing is at least a set of bug
reports. It may also include test notes, which include overall
impressions and a summary of the test strategy/thought
process. Artifacts such as test data or test materials are also both
inputs and outputs from exploratory testing.

\paragraph{Example.} I recommend reading the example by James Bach in ``et-article.pdf''
about the photo editing program (``ET in Action''). He describes how he used ET to evaluate
software for Windows compatibility and found critical defects.

\section*{In-class exercise: Exploratory testing of WaterlooWorks.}
We will try out exploratory testing with WaterlooWorks. I believe that everyone should
have access to the system, although for some of you there may be no jobs visible
right now. 

The charter will be ``Explore the overall functionality of WaterlooWorks''. Summarize
in one or two sentences what the purpose of WaterlooWorks is. Identify the tasks that
WaterlooWorks should be able to do and classify them as primary or contributing. Identify
areas of potential instability. Test each function and record results (bugs).

Of course, don't do things that have actual effects. Usually, testers would have access 
to a development server and could test those areas more aggressively. But we are working
with production systems here.


\end{document}
